\documentclass[12pt,]{article}
\usepackage{amsmath}
\usepackage{amssymb}
\usepackage{hyperref}
\usepackage{xcolor}
\usepackage{multirow}

\hypersetup{colorlinks=true,linkcolor=blue,urlcolor=blue}


\begin{document}

\section{Tutorial on Using {\tt R} Code}
This tutorial explains how to use the {\tt R} code presented in the paper: \\ \\
Skelly DA, Johansson M, Madeoy J, Wakefield J, Akey JM (2011) A powerful 
and flexible statistical framework for testing hypotheses of allele-specific gene 
expression from RNA-Seq data. \emph{Genome Res} 21:1728-37\\

\subsection{Dependencies}
Before you can use the {\tt R} scripts described below as intended, you must
install the {\tt optparse} {\tt R} package found at \url{http://cran.r-project.org/}.

\subsection{{\tt R} Scripts}
For the analysis you will need two main {\tt R} scripts and one ``accessory'' 
{\tt R} script.
\begin{enumerate}
\item {\tt DNAmodel.R} - implements the model for read counts derived from 
genomic DNA, which is used to estimate parameters for the RNA read count model.
Available at \href{https://github.com/daskelly/ase/DNA\_model/orig/DNAmodel.R}{\nolinkurl{DNA\_model/orig/DNAmodel.R}}.
\item {\tt RNAmodel.R} - implements the model for read counts derived from
cDNA, which is used to identify allele-specific expression.
Available at \href{https://github.com/daskelly/ase/RNA\_model/orig/RNAmodel.R}{\nolinkurl{RNA\_model/orig/RNAmodel.R}}.
\item {\tt readGzippedMcmcOutput.R} - script that can be {\tt source}d 
from within an {\tt R} session and is used to read the saved output of 
scripts 1 and 2. Available in each of the above directories.
\end{enumerate}

\subsection{Input format}
The read counts that must be fed into {\tt DNAmodel.R} and {\tt RNAmodel.R} should 
consist of read counts at each SNP, not aggregated by gene. The file must be in the
following tab-delimited format: \\ \\
\begin{tabular}{ | l l l l | }
\hline
\multicolumn{4}{| l |}{\# comments (ignored)} \\   % second { } is pipe, lowercase L, pipe
\multicolumn{4}{| l |}{\# next line is header which is also ignored} \\
geneNameColumn & snpIndexColumn & alleleOneCount & alleleTwoCount \\
\vdots & \vdots & \vdots & \vdots \\
\vdots & \vdots & \vdots & \vdots \\
\vdots & \vdots & \vdots & \vdots \\
\hline
\end{tabular} \\ \\
where leading lines beginning with a \# are ignored as comments,
{\tt geneNameColumn} names the first column of genes, {\tt snpIndexColumn}
is a column used to give a unique name/numerical value to each SNP (not 
currently used in the scripts), and {\tt alleleOneCount} and {\tt alleleTwoCount} are
just the read counts themselves.

\subsection{Basic Usage} \label{basic}
The scripts are designed to be used from the command-line in a unix-like environment,
but could easily be adapted for Windows usage. {\tt DNAmodel.R} and {\tt RNAmodel.R}
have command-line flags that give the basic details for usage. The arguments for
{\tt DNAmodel.R} are:
\begin{enumerate}
\item name of infile - genomic DNA read counts
\item name of outfile - the file in which to store results of MCMC. File will be gzipped.
\item number of iterations of MCMC to run
\item value of ``{\tt thin}'' - one out of every {\tt thin} iterations, the parameter values will be written to 
your outfile
\item number of scaling iterations - the proposal distributions can be scaled to achieve an
acceptance rate of (by default) approximately 30\%. The tuning will happen after blocks of
MCMC of this number of iterations. This number does not count towards the number of
MCMC iterations specified above, and the parameter values during the scaling iterations
are not saved.
\end{enumerate}
For details on the arguments to {\tt RNAmodel.R}, run the program with no arguments 
({\tt ./RNAmodel.R}). Your options are similar to those for {\tt DNAmodel.R}, but can be
specified in a slightly more flexible manner. The general look of a command to run this
program is \\
{\tt ./RNAmodel.R [options] dataset1 [dataset2 dataset3 ...] outfile}\\
where the number of input datasets can vary but must be at least one. As stated above, the
script {\tt readGzippedMcmcOutput.R} is designed to be {\tt source}d from within an {\tt R}
session and is used to read the saved output of scripts 1 and 2.

\subsection{Example Usage} \label{example}
Let's say we have collected one lane of genomic DNA read counts and four lanes of 
RNA-Seq data using the Illumina sequencing platform, and have calculated allele-specific 
read counts for each sample and collected them in the files {\tt DNA.txt}, {\tt RNA1.txt}, 
{\tt RNA2.txt}, {\tt RNA3.txt}, and {\tt RNA4.txt}. A simple analysis of this data might be:
\begin{enumerate}
\item First, run an analysis of your genomic DNA data: \\
{\tt ./DNAmodel.R DNA.txt DNAresults.gz 200000 40 2000}
\item Next, analyze the results of step 1 and obtain estimates of $\hat{a}$ and $\hat{d}$. 
\emph{Note the warning below about checking for convergence}. Inside an {\tt R} session,
you could execute the commands: \\
{\tt > source(`readGzippedMcmcOutput.R') \\
> result <- read.mcmc(`DNAresults.gz') \\
> n.iter <- result\$n.iter/result\$thin \\
> burnin <- 0.1*n.iter \\
> a.hat <- median(exp(result\$mcmc\$logA[burnin:n.iter])) \\
> d.hat <- median(exp(result\$mcmc\$logD[burnin:n.iter])) \\
}
\item Run an analysis on your RNA data: \\
{\tt ./RNAmodel.R --n.iter=500000 --thin=100 --n.scaling.iter=2000 --a.hat=2000 --d.hat=500 
--max.rounds.of.scaling=8 RNA1.txt RNA2.txt RNA3.txt RNA4.txt RNAresults.gz} \\
\emph{Note the warning below about checking for convergence}.
\item Examine the results of the RNA run in {\tt R}: \\
{\tt > source(`readGzippedMcmcOutput.R') \\
> result <- read.mcmc(`RNAresults.gz') \\
}
\end{enumerate}

\subsection{Extracting results}
After you run the scripts as described above in sections \ref{basic}-\ref{example}, you 
will end up with a gzipped file of MCMC results (in section \ref{example}, this is named
{\tt RNAresults.gz}). From this file, the most useful quantities that can be obtained are
the variables $\pi_0$; $p_i$ and $e_i$ for each gene $i$; and the probability that gene $i$
shows allele-specific expression. For this latter quantity we will need to use the formula
at the bottom of page 9 in the supplementary material, which requires values of $p$, $e$, 
$f$, $g$, $h$, $\pi_0$, $\hat{a}$, and $\hat{d}$.

To obtain estimates of these quantities we can execute the following commands inside 
of an {\tt R} session: \\
{\tt > source(`readGzippedMcmcOutput.R') \\
\# define the inverse logit function \\
> inv.logit <- function(val) exp(val)/(1 + exp(val)) \\
> result <- read.mcmc(`DNAresults.gz') \\
> genes <- result\$features \qquad \# character vector of gene names\\
> a.hat <- result\$a.hat \\
> d.hat <- result\$d.hat \\
\# below we take the posterior median of MCMC samples that are vectors of length 
equal to n.iter \\
> pi0 <- median(inv.logit(result\$mcmc\$logitPi0)) \\
> f <- median(exp(result\$mcmc\$logF)) \\
> g <- median(exp(result\$mcmc\$logG)) \\
> h <- median(exp(result\$mcmc\$logH)) \\
\# below we take the posterior median across columns of matrices with nrow==n.iter and 
ncol==length(genes) \\
> p <- apply(inv.logit(result\$logit\_p), 2, median) \\
> e <- apply(inv.logit(result\$logit\_e), 2, median)
} \\

Now we have all the quantities we need. {\tt p} and {\tt e} are vectors of length equal to
the number of genes, so to get estimates of (say) $p_{17}$ and $e_{17}$ we just look at
{\tt p[17]} and {\tt e[17]} in {\tt R}. To get the posterior probability of ASE for a particular gene
we can use the following formula (reproduced from page 9 in the supplement)
\begin{align*}
p &= \frac{\mbox{\tt dbeta}(p | f, g) \mbox{\tt dbeta}(e | 1, h) (1 - \pi_0)}
	{\mbox{\tt dbeta}(p | f, g) \mbox{\tt dbeta}(e | 1, h) (1 - \pi_0) + \mbox{\tt dbeta}(p | \hat{a}) \mbox{\tt dbeta}(e | 1, \hat{d}) \pi_0}
\end{align*}
where {\tt dbeta} is the beta density function.

We might also be interested in calculating the false discovery rate when calling a set of 
features significant. Say we have used the above formula to calculate the posterior
probability of ASE for every gene (and we have stored this result in a numeric vector 
called {\tt posteriorProbAse} (of length equal to the number of genes for which we
measured expression levels). We can then calculate the overall false discovery rate
when calling features with (say) posterior probability of ASE $> 0.8$ as significant with the 
following
{\tt R} code: \\
{\tt > cutoff <- 0.8 \qquad \# change cutoff at will\\
> calledSignificant <- which(posteriorProbAse >= cutoff) \\
> fdr <- mean((1 - posteriorProbAse)[calledSignificant]) \\
} \\

\subsection{Sample data}
I have compiled a small sample RNA dataset that you may wish to use to ensure that the {\tt R}
scripts are running properly. This dataset consists entirely of simulated data so the true
values of all parameters are known. The dataset is 50 genes, where each gene has between
1 and 5 SNPs and the coverage at each SNP is Poisson distributed with mean 300. In this
simulated data, half of the genes show ASE, and half do not show ASE. The following are the 
true values of each parameter:
\begin{align*}
\hat{a} &= 500 \\
\hat{d} &= 1000 \\
f &= 5 \\
g &= 5 \\
h &= 20 \\
\pi_0 &= 0.5
\end{align*}
Note: these values have been chosen rather arbitrarily, so it is not necessarily bad if you see
values that are quite different in your actual (non-simulated) data.
I ran {\tt RNAmodel.R} using this simulated data with the command:\\
{\tt ./RNAmodel.R --n.iter=100000 --thin=100 --n.scaling.iter=2000 \\
--max.rounds.of.scaling=5 --a.hat=500 --d.hat=1000  simulatedData50genes.txt 
\\ simulatedData50genes-out.gz 2> simulatedData50genes1.log}\\
This took about 25 minutes to run on a 3.33 GHz CPU with 4 Gb RAM.
Executing the following code in an {\tt R} session shows that the results are as we
expect, i.e. the posterior distributions of parameters include their true values and inferred
values of $p_i$ and $e_i$ for each gene are close to their true values. \\
{\tt  > source(`readGzippedMcmcOutput.R') \\
> inv.logit <- function(val) exp(val)/(1 + exp(val)) \\
> result <- read.mcmc(`simulatedData50genes-out.gz') \\
> hist(exp(result\$mcmc\$logF)) \\
> hist(exp(result\$mcmc\$logG)) \\
> hist(exp(result\$mcmc\$logH)) \\
> hist(inv.logit(result\$mcmc\$logitPi0)) \\
> load(`trueValues.Rdata') \\
> p.true \\
\quad [1] 0.6626116 0.5060690 0.4690048 0.6610554 0.3463875 0.5158258 \\
\quad [7] 0.5174206 0.5130136 0.4984910 0.5023504 0.5410948 0.5175285 \\
\quad [13] 0.5137415 0.5620388 0.5132943 0.1974173 0.5578723 0.6564482 \\ 
\quad [19] 0.3839517 0.4930944 0.7539602 0.4922985 0.6184599 0.4791786 \\
\quad [25] 0.5062900 0.4797017 0.5343286 0.5175784 0.4979721 0.3066247 \\
\quad [31] 0.7620101 0.5146986 0.4837929 0.5209062 0.4610464 0.7443861 \\
\quad [37] 0.5376732 0.3976954 0.5138220 0.4767660 0.6537073 0.5078267 \\
\quad [43] 0.2693251 0.4760823 0.6723698 0.6128043 0.7393318 0.5129163 \\
\quad [49] 0.5222207 0.7043429 \\
> e.true \\
\quad [1] 2.280807e-02 1.613213e-03 5.445363e-04 5.830355e-04 9.550718e-03 \\
\quad [6] 1.218790e-02 5.801549e-04 3.012189e-04 7.039235e-04 7.168069e-04 \\
\quad [11] 6.174186e-02 1.266529e-03 2.819844e-03 3.346149e-02 2.820236e-04 \\
\quad [16] 8.239054e-02 2.550545e-03 5.751159e-02 3.241195e-02 4.889295e-04 \\ 
\quad [21] 1.601494e-01 2.486705e-04 3.670602e-02 2.741677e-03 1.286787e-03 \\ 
\quad [26] 1.949170e-03 1.174512e-01 2.276314e-04 1.597160e-03 3.791292e-02 \\
\quad [31] 1.011558e-01 1.179732e-03 2.870782e-03 4.076936e-04 1.253305e-03 \\
\quad [36] 1.305254e-01 2.195357e-04 6.520508e-02 3.930946e-04 1.557770e-04 \\
\quad [41] 2.102296e-01 7.946806e-05 5.680259e-02 4.817029e-04 4.261961e-02 \\
\quad [46] 3.001975e-02 1.378645e-01 3.362148e-04 5.414582e-04 1.167085e-02 \\
> p.mcmc <- inv.logit(apply(result\$logit\_p, 2, median)) \\
> e.mcmc <- inv.logit(apply(result\$logit\_e, 2, median)) \\
> cor(p.true, p.mcmc) \\
\quad [1] 0.9562227 \\
> cor(e.true, e.mcmc) \\
\quad [1] 0.7886158 \\
}
I ran the MCMC for much longer using the following command: \\
{\tt ./RNAmodel.R --n.iter=1000000 --thin=1000 --n.scaling.iter=5000 \\
--max.rounds.of.scaling=5 --a.hat=500 --d.hat=1000  simulatedData50genes.txt 
\\ simulatedData50genes-out.gz 2> simulatedData50genes1.log}\\
This took slightly under four hours to run on a 3.33 GHz CPU with 4 Gb RAM. The
results were in very close agreement with the results for the shorter run above.
A discussion of the use of statistical methods for diagnosing convergence of
MCMC is beyond the scope of this document, but the references below may
be helpful for this task.


\subsection{Warning!}
MCMC can be dangerous! Make sure you understand something about MCMC before
attempting to run these scripts. You should always run multiple chains from different
starting parameters, and verify convergence by examining time series plots or by other
more formal measures such as the convergence diagnostics proposed by Gelman and 
Rubin \cite{gelman_inference_1992}, 
Geweke \cite{geweke_evaluating_1992}, 
or Heidelberger and Welch 
\cite{heidelberger_simulation_1983}. 
These scripts do nothing to check that you have 
completed any of these tasks.

\bibliographystyle{GenomeBiology}
\bibliography{tutorial.bib}

\end{document}
